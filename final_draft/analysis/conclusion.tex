\subsection{Conclusion}
The EPA's engineers acted as software engineers by conducting tests on portions of VW cars controlled by the cars' software\cite{se_code, epa_track}. The EPA's engineers are therefore subject to the SE Code. Through an analysis of SE Code Tenets 1.03, 1.05, and 3.01, the EPA's engineers have been shown to be in compliance with the SE Code.

Analysis of Tenet 1.03 shows that although the EPA's engineers failed to detect VW's defeat device\cite{study} and ultimately allowed for excessive emissions of Nitrogen Oxide\cite{emissions_standards}, their reasoning was based in excellent judgment and reasoning\cite{anab_accreditation}. Additionally, the EPA's engineers were shown to be not at fault for the flaws in test design that allowed VW to cheat, as the engineers were required to implement components of the test a certain way to maintain transparency to the public\cite{emissions_standards}.

Analysis of Tenet 1.05 further backs up the EPA's engineers' decision to allow the publishing of their test track. The EPA's engineers fulfill both the component of the tenet pertaining to their ability to seriously test VW's software\cite{anab_accreditation} as well as the component supporting the importance of drawing the attention of the American people to any wrongdoing discovered when testing\cite{epa_ghg_report}.

Analysis of Tenet 3.01 also deals with the EPA's maintenance of transparency with the American people. The analysis of this tenet solidifies the idea that maintenance of transparency is essential to remaining compliant with the SE Code\cite{epa_track_details}.

The analysis of these tenets demonstrates that the EPA's engineers were not in violation of the software engineering code, as their actions were consistently in compliance with the tenets selected.