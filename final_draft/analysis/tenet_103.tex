\subsection{Tenet 1.03}
\subsubsection{Tenet}
\boxed{
    \ul{Approve software} only if they have a \ul{well-founded belief} that it is \ul{safe}, \ul{meets specifications}, passes \ul{appropriate tests}, and does not diminish \ul{quality of life}, diminish privacy or \ul{harm the environment}. The ultimate effect of the work should be to \ul{the public good}.
}

\level{4}{Definitions}
\level{5}{Approve Software}
To approve is defined as ``to confirm or sanction formally''\cite{dictionary_com}. Software is defined as ``the programs used to direct the operation of a computer,...''\cite{dictionary_com}. Therefore, a general definition of approve software is ``to confirm or sanction formally the programs used to direct the operation of a computer''.

%%%% NOTE: HARDCODED FOOTNOTE. VERIFY BEFORE SUBMISSION. %%%%
As defined above\footnote{5.1.1.8.1}, the computers being tested were those running on VW's cars. Therefore, a domain specific definition of software approval is ``to confirm or sanction formally the software controlling VW's cars''.

\level{5}{Well-Founded Belief}
A Well-Founded Belief is defined as a belief ``that is based on excellent reasoning, information, judgment, or grounds''\cite{merriam_webster}.

\level{5}{Safe}
Safe is defined as ``secure from liability to harm, injury, danger, or risk''\cite{dictionary_com}.
% In the context of this case, safety is attained when Nitrogen Oxide emissions are at or below the amount legally allowed by the government.

\level{5}{Meets Specifications}
To meet is defined as ``to come upon; come into the presence of; encounter''. Specifications are defined as ``a detailed description or assessment of requirements,..., materials, etc., as of a ... machine, ...''\cite{dictionary_com}. Therefore, to say a machine meets specifications is to say that the design has ``come into the presence of'', or is aligned with, ``a detailed description or assessment of requirements, materials, etc.''.
% In the context of this case, specifications are considered met when the automobile passes the EPA emissions test.

\level{5}{Appropriate Tests}
Appropriate is defined as ``suitable or fitting for a particular purpose''\cite{dictionary_com}. A test is defined as ``the trial of the quality of something''\cite{dictionary_com}. Therefore, appropriate tests can be defined as ``trials suited to a particular purpose that measure the quality of something''.
% Tests refer to the EPA emissions tests. Appropriate tests are those that are considered to adequately assess the emissions of a vehicle.

\level{5}{Quality of Life}
Quality of Life\cite{quality_of_life} refers to a combination of basic human needs (food, shelter, etc.) and subjective well-being. Subjective well-being refers to individuals' or groups' perceived levels of happiness, satisfaction, etc. in regards to the degree to which their basic human needs are met.

In this case, Quality of Life will more specifically refer to the health-related aspects of basic human needs\cite{quality_of_life}.

\level{5}{Harm the Environment}
Harm is defined as ``to do or cause harm to; injure; damage; hurt''\cite{dictionary_com}. The environment is defined as ``the air, water, minerals, organisms, and all other external factors surrounding and affecting a given organism at any time''\cite{dictionary_com}. Therefore, to harm the environment can be defined as ``to damage the air, water, minerals, organisms, and all other external factors surrounding and affecting a given organism at any time''.

\level{5}{The Public Good}
The public good is defined as ``the well-being of the general public''\cite{dictionary_com}. Well-Being is defined as ``a state characterized by health, happiness, and prosperity''\cite{dictionary_com}. General public is defined as ``all the people in an area, country, etc.''\cite{merriam_webster}. The public good can therefore be defined as ``the health, happiness, and prosperity of all of the people in an area, country, etc.''.

While no cars have zero environmental impact\cite{tesla_env_impact}, there is a legal limit to the amount of emissions a vehicle produces before it is deemed work against the public good\cite{emissions_standards}. As defined above, the software systems being evaluated are those running the operations of VW's cars. A domain specific definition for the public good is ``ensuring VW's cars produce emissions that conform to the governing standards for emissions''.

\level{4}{Domain Specific Tenet 1.03}

\boxed{
    An EPA engineer should \ul{formally sanction the software controlling VW's cars} only if they have a \ul{belief based on excellent reasoning, information, judgment, or grounds} that it is \ul{secure from liability to harm, injury, danger, or risk}, \ul{is aligned with a detailed description or assessment of requirements, materials, etc.}, passes \ul{trials suited to a particular purpose that measure the quality of the software}, and does not diminish \ul{the health-related aspects of human needs}, diminish privacy or \ul{damage the air, water, minerals, organisms, or any other external factors surrounding and affecting a given organism at any time}. The ultimate effect of the work should be to \ul{ensuring VW's cars produce emissions that conform to the governing standards for emissions}.
}

\level{4}{Analysis}
\level{5}{Excellent Reasoning, Information, Judgment, or Grounds}
The judgment and reasoning exercised by EPA engineers when designing and conducting emissions tests has been validated\cite{anab_accreditation} by ANAB\cite{anab_recognition}, a body designed to accredit labs.

\level{5}{Alignment with Requirements}
VW's cars would be considered in alignment with requirements by the EPA due to the defeat device used\cite{study_details}. In reality, VW's cars were not in alignment with the requirements, as they would not have passed the EPA's tests without cheating\cite{study}.

\level{5}{Trials to Measure Quality}
The EPA's engineers had designed tests to measure the levels of several types of emissions, including Nitrogen Oxide\cite{epa_track_details}. These tests are meant to simulate real-world driving conditions in a number of common scenarios. However, the EPA's engineers were required to make the testing information public before actually conducting the tests\cite{study_details}. 

\level{5}{Health-Related Aspects of Human Needs}
The West Virginia University study shows that VW's automobiles were emitting Nitrogen Oxide at levels of almost 40 times those allowed by the EPA\cite{study_details}. Nitrogen Oxide poses a serious risk to human health, and mainly affects the respiratory system\cite{health_effects_of_no}. It is clear that the EPA's engineers allowed the production of software that diminished the health-related aspects of human needs.

\level{5}{Emissions Conform to Standards}
Although VW's cars appeared to conform to standards\cite{study_details}, they did not\cite{study}. Therefore, the EPA's engineers ultimately failed to ensure that VW's cars produced emissions that conformed to the governing standards for emissions.

\level{4}{Conclusion}
Although the EPA's engineers (1) failed to ensure that VW's cars produced emissions that conformed to the governing standards for emissions and (2) allowed the production of software that diminished the health-related aspects of human needs, there are multiple qualifiers that complicate the determination of whether the EPA's engineers violated the Software Engineering Code.

The EPA's engineers chose to approve the emissions levels of VW's automobiles based on excellent judgment and reasoning. The EPA emissions testing laboratory has been accredited\cite{anab_accreditation} by ANAB\cite{anab_recognition}, and is therefore considered to be capable of adequately designing and implementing labs.

The EPA's engineers verified that VW's cars did conform to emissions standards in their tests\cite{study}. However, these results were false due to deception on the part of VW\cite{course_detection}. 

The EPA's engineers verified VW's cars' emissions via a standard test\cite{epa_track}. While the test failed to correctly assess the real-world emissions of VW's vehicles, the fault is not entirely on the EPA's engineers. VW was able to exploit the testing process by gaining an understanding of the test track, which the EPA is required to publish\cite{epa_track_details}. Therefore, VW was able to cheat the EPA emissions test due to a systemic oversight on the part of the EPA, rather than a flawed design or implementation on the part of the engineer.

Because the EPA's engineers made their assessment based on excellent judgment and reasoning, verified that VW's cars passed their test, and were not at fault for VW's exploitation of the test, they meet the qualifiers showing that they were not in violation of the Software Engineering Code of Ethics.